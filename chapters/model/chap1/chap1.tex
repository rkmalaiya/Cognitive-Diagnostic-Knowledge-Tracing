\chapter{Mathematical Model}

\begin{shadowbox}{\begin{smlpage}{Objectives}
\begin{large}
\begin{smlitemize}
\begin{item}
Statistical Environment
\end{item}
\begin{item}
Probability Model
\end{item}
\begin{item}
Objective Function
\end{item}
\begin{item}
Derivatives of Objective Function
\end{item}
\begin{item}
Parameter Estimation
\end{item}
\end{smlitemize}
\end{large}
\end{smlpage}}\end{shadowbox} \\

\section{Statistical Environment}


\begin{definition}[Student Response]

\begin{multline}
    {\bf X}^{i}(t)  = \left[ 
\left[\bfs{x}^{i}_{1}(t), \bfs{m}^{i}_{1}(t) \right],
\left[\bfs{x}^{i}_{2}(t), \bfs{m}^{i}_{2}(t) \right]...
\left[\bfs{x}^{i}_{J}(t), \bfs{m}^{i}_{J}(t) \right]
\right]
\\ \text{and} \\
    \bfs{x}^{i}_{j}(t) = \begin{cases}
    \vec{0}, & \text{if student skips $j^{th}$ question}.\\
    \bfs{\zeta}^{d(j)}_{z}, & \text{if student selects $z^{th}$ answer combination for $j^{th}$ question}.
  \end{cases} 
\end{multline}

  where $\bfs{\zeta}^{d(j)}_{z}$ is $z^{th}$ column from an identity matrix of $d(j)$ dimension and $\bfs{\zeta}^{d(j)}_{z'}$ represents ideal answer choice

\begin{displaymath} 
  m^{i}_{j}(t)=\begin{cases}
    1, & \text{if $i^{th}$ student attempts $j^{th}$ question}.\\
    0, & \text{otherwise}.
  \end{cases}
\end{displaymath}
\begin{displaymath} 
  \bfs{m}^{i}(t)= \vec{0} \text{ if $i^{th}$ student misses the $t^{th}$ assessment period}
\end{displaymath}
Example: Student responses can be represented as:
\[
\bfs{x}^1_3 = \{0, 0, 0, 1, 0, 0\} \text{if student chose $z=4$ answer combination, where $j=3$, $d(3) = 6$ }\\
\]
\end{definition}

\begin{definition}[$\bfs{Q}$ Matrix]
Let $\bfs{Q}$ matrix be defined such that

\begin{equation*}
 \bfs{Q} = [ {\bfs{q}_1, \bfs{q}_2, ..., \bfs{q}_J}]
\end{equation*}
where 
\begin{equation*}
    \bfs{q}_{j} \in \{ 0,1 \}^{d(j) \times K}
\end{equation*}

\begin{equation}
  q_{jd(j)k}=\begin{cases}
    1, & \text{ if $ k^{th} $ skill is present for $d(j)^{th}$ response. }\\
    0, & \text{otherwise.}
  \end{cases} 
\end{equation}


\end{definition}

\begin{definition}[ $\bfs{U}$ Matrix]
Let $\bfs{U}  \in \{ 0,1 \}^{K \times K}$ be a matrix defined such that

\begin{equation}
  \bfs{u}_{mk}=\begin{cases}
    1, & \text{ if mastery in $ k^{th} $ learning objective influences } \\
    & \text{ probability of mastery in $m^{th}$ learning objective in furture.   }\\
    0, & \text{otherwise.}
  \end{cases}
\end{equation}
here we assume that a single learning objective requires only single latent skill
\end{definition}

\begin{definition}[Student Situation Model]
\begin{equation}
\bfs{\alpha}^{i}(t) \equiv [\alpha^{i}_{1}(t), \alpha^{i}_{2}(t)...\alpha^{i}_{K}(t) ]
\end{equation}

$\bfs{\alpha}^{i}(t)$ represents $i^{th}$ student's situation model during $t^{th}$ assessment period
\begin{equation}
  \alpha^{i}_{k}=\begin{cases}
    1, & \text{ $\iff$ $ k^{th} $ latent skill is present in $i^{th}$ student's} \\
    & \text{  situation model at $t$ time period assessment }\\
    0, & \text{otherwise.}
  \end{cases}
\end{equation}
\end{definition}

\section{Probability Model}
\begin{definition}[Stochastic Sequence]

\begin{equation}
    \left ( \bfs{X}^{i}(t), \bfs{m}^{i}(t), \bfs{\alpha}^{i}(t) \right )_{t=0}^{T} 
\end{equation}

\end{definition}

\begin{definition}[Emission Probability Model]

\begin{equation}
    P(\bfs{x}^{i}_{j} | \bfs{\alpha^{i}}, \bfs{\beta}_{j}) = \phi \left (  \beta_{_{jR}} \eta^{i}_{j}(t) + \beta_{_{jR}} \left ( 1 - \eta^{i}_{j}(t) \right ) \right )
\end{equation}
where 
\begin{equation*}
    \eta^{i}_{j}(t) = \prod_{k=1}^{K} \left ( \alpha^{i}_{k} (t) \right )^{q_{_{jd(j)k}}}
\end{equation*}
\begin{equation*}
    \bfs{\beta}_{j} = \left[ \beta_{_{jR}}, \beta_{_{jG}} \right]^{T}
\end{equation*}
\begin{equation*}
    \phi(\psi) = \left ( 1 + \exp(-\psi) \right )^{-1}
\end{equation*}
\end{definition}

\begin{definition}[Transition Probability Model]
\begin{equation*}
    P \left ( \bfs{\alpha^{i} (t+1)} | \bfs{\alpha^{i} (t)}, \bfs{\theta}^{i}  \right ) =
\end{equation*}
\begin{equation}
    \prod^{K}_{k=1} 
    \left ( P \left ( \alpha^{i}_{k} (t+1) = 1 | \bfs{\alpha}^{i} (t), \bfs{\theta}^{i}  \right )^{\alpha^{i}_{k}(t+1)}
    P \left ( \alpha (t+1)^{i}_{k} = 0 | \bfs{\alpha}^{i} (t), \bfs{\theta}^{i}  \right )^{ \left(1 - \alpha^{i}_{k}(t+1)\right )} 
    \right )
\end{equation}
where 

\begin{equation*}
     P \left ( \alpha^{i}_{m} (t+1) = 1 | \bfs{\alpha}^{i} (t), \bfs{\theta}^{i}  \right ) =
     \phi\left ( \theta_{i,m,0} + \theta_{i,m,m}\alpha_{im}(t) + \sum_{k=1,k \neq m}^{K} \theta_{i,m,k}u_{m,k}\alpha_{ik}(t) \right )
\end{equation*}

    $\theta^{i} \in \mathbb{R}^{K\times K} $

    $\theta_{i,m,0}$ is student-specific course learning objective mastery parameter. Specifies level of difficulty??

    $\theta_{i,m,k}$ measures the likelihood the student will demonstrate mastery of the $m^{th}$ learning objective latent skill given the student is currently demonstrating mastery of the $k^{th}$ learning objective latent skill.
    
    $\theta_{i,m,m}$ is student-specific situation model momentum parameter. It captures likelihood that student situation model will change when learning process evolves from $t$ to $(t+1)$
    
    For CDKT, $\theta_{i,m,m} = \theta_{i,k,k} \forall k,m = {1,...,K}$ which indicates that each latent skill has equal likelihood of transitioning from $t$ to $(t+1)$
\end{definition}








